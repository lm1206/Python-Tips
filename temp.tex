\documentclass{ctexart} % Default font size is 12pt, it can be changed here
\ctexset{section/format = {\Large\bfseries}}
\usepackage{geometry} % Required to change the page size to A4
\geometry{a4paper} % Set the page size to be A4 as opposed to the default US Letter
\geometry{left=1.6cm,right=1.6cm,top=3.5cm,bottom=3.5cm}
%\usepacjage{subfigure}
\usepackage{graphicx} % Required for including pictures
\usepackage{amsmath}
\usepackage{amssymb,bm}
\usepackage{enumerate}
\begin{document}
 假设 5个人分别为: $x_1,x_2,x_3,x_4,x_5$\\
 假设5天分别为:  $y_1,y_2,y_3,y_4,y_5$\\
则我用 一个矩阵表示他们的考勤安排:
\begin{equation}
\begin{array}{cccccc}
 &y_1&y_2&y_3&y_4&y_5\\
x_1&1&1&0&0&0\\
x_2&0&1&1&0&0\\
x_3&0&0&1&1&0\\
x_4&0&0&0&1&1\\
x_5&1&0&0&0&1
\end{array}
\end{equation}
1表示值班 ,0表示休息,显然矩阵中必须有 10个1,15个0\\
需要满足的三个条件为: 
\begin{itemize}
\item  1. 每一行相加等于2 
\item  2. 每一列相加等于2
\item  3. 行列式值非0
\end{itemize}


很显然,任意对调两行或者两列,得到的新的矩阵依然满足以上的条件,也就意味着一种排班方法。我把上面那种排班方法记为
\begin{equation}
[x_1,x_2,x_3,x_4,x_5]\times[y_1,y_2,y_3,y_4,y_5]
\end{equation}
都对应着一种排班 方法。\\

假设存在一种排班方法:$[x‘_1,x’_2,x‘_3,x’_4,x‘_5]\times[y‘_1,y’_2,y‘_3,y’_4,y‘_5]$
\begin{enumerate}[(1)]
\item 做映射$y_1\rightarrow y'_1$
\item 将$y_1$中的两个员工分别映射成$x_1\rightarrow x'_1$和$x_2\rightarrow x'_2$
\item 将$x_1$值班的另一天分映射成$y'_2$(假设为$y_2$),一起值班的人映射成$x'_5$(假设为$x_5$),
\item 将$x_2$值班的另一天分映射成$y'_3$(假设为$y_3$),一起值班的人映射成给$x'_3$(假设为$x_3$),
\item 将$x_5$值班的另一天分映射成$y'_5$(假设为$y_5$),一起值班的人映射成$x'_4$(假设为$x_4$),
\item 将$x_3$值班的另一天分映射成$y'_4$(只能是$y_4$)
\end{enumerate}
按以上的方案,$[x_1,x_2,x_3,x_4,x_5]\times[y_1,y_2,y_3,y_4,y_5]$就可以和$[x‘_1,x’_2,x‘_3,x’_4,x‘_5]\times[y‘_1,y’_2,y‘_3,y’_4,y‘_5]$完成一个映射关系。

\end{document}